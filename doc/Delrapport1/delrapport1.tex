\documentclass[12pt]{article}
\usepackage{amsmath} % flere matematikkommandoer
\usepackage{amssymb}
\usepackage[utf8]{inputenc} % æøå
\usepackage[T1]{fontenc} % mere æøå
\usepackage[danish]{babel} % orddeling
\usepackage{verbatim} % så man kan skrive ren tekst
\usepackage[a4paper, margin = 1in]{geometry}
\usepackage{graphicx}

\title{Delrapport 1}
\author{Christian Kjær Larsen, Lukas Svarre Engedal, Tobias Sønderskov Hansen}

\begin{document}
\maketitle

\section{Problemformulering}
\subsection{Problemområde}
Fra næste år af laver de markant om på datalogi uddannelsen, og i den forbindelse slår de bl.a. kurserne \emph{Introduktion til Programmering} og \emph{Objektorienteret programmering og design} sammen til et nyt kursus, hvor de i stedet for SML og Java vil bruge F\#. I den forbindelse forudser kursuslederne at det muligvis kan blive svært at finde tilstrækkeligt med instruktorer til at hjælpe de studerende med at lære F\# og med at løse opgaver i F\#, og kursuslederne vil derfor have automatiseret hele opgaveaspektet lidt. 
I den forbindelse har de bedt os om at hjælpe dem med at lave et slags interaktivt lærinssystem, hvor de studerende kan logge ind og løse en række opgaver der skal hjælpe dem med at lære F\#, de kan få hints til opgaverne hvis de har brug for det, kursuslederne kan tilføje nye og  ændre eller fjerne eksisterende spørgsmål, og desuden få en slags statistik over hvordan de studerende klarer opgaverne.
\section{Indledende projektplan}
\section{System og softwarearkitektur}
\section{Projektaftalen}
\section{Intern projektetablering}

\end{document}
