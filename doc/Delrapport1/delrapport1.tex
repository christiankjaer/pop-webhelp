\documentclass[12pt]{article}
\usepackage{amsmath} % flere matematikkommandoer
\usepackage{amssymb}
\usepackage[utf8]{inputenc} % æøå
\usepackage[T1]{fontenc} % mere æøå
\usepackage[danish]{babel} % orddeling
\usepackage{verbatim} % så man kan skrive ren tekst
\usepackage[a4paper, margin = 1in]{geometry}
\usepackage{graphicx}

\title{Delrapport 1}
\author{Christian Kjær Larsen, Lukas Svarre Engedal, Tobias Sønderskov Hansen}

\begin{document}
\maketitle

\section{Problemformulering}
\subsection{Scenarier}
\paragraph{Tilføjelse af spørgsmål}
En underviser på kurset går ind på hjemmesiden. Vedkommende logger ind på siden, og han tages til en speciel side for administatorer.
Her skal han tilføje en ny opgave. Vedkommende giver spørgsmålet en kategori efter hvilket læringsmål det opfylder. Opgavesteksten og de rigtige svar skrives ind. Eventuelt tilføjes hints.

\paragraph{Løsning af opgaver}
En studerende på kurset logger ind på hjemmesiden, vedkommende tages til en side, hvor man ser alle læringsmålene, og hvad man allerede har opfyldt. Den studerende vælger et af de tilgængelige læringsmål, og en side åbnes hvor man kan besvare en ny opgave inden for emnet. Den studerende svarer på spørgsmålet, og får feedback med det samme, og der er mulighed for hints, hvis spørgsmålet er svært.
\end{itemize}
\section{Indledende projektplan}
\section{System og softwarearkitektur}
\section{Projektaftalen}
\section{Intern projektetablering}

\end{document}
