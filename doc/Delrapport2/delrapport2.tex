\documentclass[12pt]{article}
\usepackage{natbib}
\usepackage{eso-pic}
\usepackage{amsmath} % flere matematikkommandoer
\usepackage{amssymb}
\usepackage[utf8]{inputenc} % æøå
\usepackage[T1]{fontenc} % mere æøå
\usepackage[danish]{babel} % orddeling
\usepackage{verbatim} % så man kan skrive ren tekst
\usepackage[a4paper, margin = 1in]{geometry}
\usepackage{graphicx}
\usepackage{booktabs}
\usepackage{enumitem}
\usepackage{placeins}
\author{
  Christian Kjær Larsen\\
  \texttt{011292} \\[.4cm]
  Lukas Svarre Engedal\\
  \texttt{210790} \\[.4cm]
  Tobias Sønderskov Hansen\\
  \texttt{243095} \\[.4cm]
  Instruktor: Aske Mottelson\\[.4cm]
  \vspace{10cm}
}

\title{
  \vspace{3cm}
  \Huge{ProjDat 2015} \\[.25cm]
  \large{Delrapport 2}
}

\begin{document}

\AddToShipoutPicture*{\put(0,0){\includegraphics*[viewport=0 0 700 600]{includes/nat-farve}}}
\AddToShipoutPicture*{\put(0,602){\includegraphics*[viewport=0 600 700 1600]{includes/nat-farve}}}

%% Change `ku-en` to `nat-en` to use the `Faculty of Science` header
\AddToShipoutPicture*{\put(0,0){\includegraphics*{includes/nat-en}}}

\clearpage\maketitle
\thispagestyle{empty}

\newpage

\tableofcontents %generate table of content

\thispagestyle{empty}

\newpage
\pagestyle{plain}
\setcounter{page}{1}
\pagenumbering{arabic}

\section{Abstract}
\label{sec:abstract}

\section{Formål og rammer}
\label{sec:formal_og_rammer}
\section{Kravspecifikation}
\label{sec:kravspecifikation}

\subsection{Krav}
\label{sub:krav}
\paragraph{Funktionelle krav}
De funktionelle krav for systemet er i tråd med afsnit 4.3.1 i \cite{OOSE}, nemlig at de omhandler den specifikke brug af systemet.
\begin{itemize}
    \item Systemet skal være en hjemmeside med opgaver, som kan løses individuelt af studerende.
    \item Der kræves login, så man kan følge med i den studerendes udvikling.
    \item Spørgsmålene skal være grupperede efter hvilke læringsmål de tester.
    \item Læringsmålene skal igen grupperes efter et \emph{threshold}, som er et øvre læringsmål hvorefter der kommer sværere stof.
    \item Der skal være hints til opgaverne, og der skal kunne gives feedback til hints.
    \item Alle forsøg på opgavebesvarelser skal gemmes i en log.
\end{itemize}
Hvis der er tid i en af de senere iterationer, så er der en række ekstra krav som kan implementeres.
\begin{itemize}
    \item Loggen skal bruges til at give underviseren information om hvilke spørgsmål der er svære.
    \item Der skal tilføjes en grad af \emph{gamification}, så der gives badges og point for fremskridt, og det bliver muligt at følge med i andres fremskridt.
    \item Hvis man ikke har øvet sig i et emne i et stykke tid, så falder ens erfaring i området.
\end{itemize}

\paragraph{Ikke-funktionelle krav}
\label{par:ikke_funktionelle_krav}
I afsnit 4.3.2 i \cite{OOSE} beskrives ikke-funktionelle krav som krav, der ikke direkte beskriver funktionaliteten, men mere generelle krav omkring ting som brugervenlighed, ydeevne, pålidelighed og hvor vedligeholdesesvenligt systemet er. De ikke-funktionelle krav til dette projekt er forholdsvis begrænsede, idet vi har fået ret stor frihed i forhold til hvordan vi løser opgaven af vores kunde.

Det vigtigste krav er at det skal være nemt at tilføje nye opgaver til systemet, sådan at kursuslederne efterfølgende kan opbygge en tilstrækkelig samling af opgaver til de studerende. Derudover er det også et mere overordnet krav at det endelige produkt gerne skal være så simpelt og modulært som muligt, sådan at det er nemt at overtage, udvide og bygge videre på efter at vi overgiver projektet til kunden. Ingen af kunderne er web-udviklere, så det er også vigtigt at det valgte framework er veldokumenteret og rimeligt simpelt at bruge.


\subsection{Use cases}
\label{sub:use_cases}

\begin{figure}[htpb]
    \centering
    \begin{tabular}{r p{10cm}}
        \toprule
        \textit{Navn på use-case:} & \verb!OpretBruger! \\
        \hline
        \textit{Deltagende aktører:} & Påbegyndt af en studerende \\
        \hline
        \textit{Hændelser:} & \begin{enumerate}[nolistsep]
            \item En studerende åbner hjemmesiden og klikker på \verb!register!
            \item Den studerende indtaster sine brugeroplysninger, dvs. ku-id og løsen.
            \item Systemet opretter brugeren, og sender en e-mail med et aktiveringslink.
            \item Den studerende klikker på linket, og kontoen aktiveres.
            \item Den studerende bringes til login-siden.
            \item Den studerende indtaster ku-id og løsen og klikker \verb!Login!
            \item Der informeres om succesfuldt login, og brugeren bringes til applikationen.
        \end{enumerate}  \\
        \hline
        \textit{Startbetingelse:} & En studerende har ingen konto, og er ikke logget ind. \\
        \hline
        \textit{Slutbetingelse:} & En studerende har en aktiveret konto og er logget ind. \\
        \bottomrule
    \end{tabular}
    \caption{Use case omkring opretning af brugere}
    \label{fig:use_case1}
\end{figure}

\subsection{Klassediagram}
Blah

\subsection{BCE-model}
Blah

\subsection{Sekvens-diagrammer}
Blah

\section{Systemdesign}
\label{sec:systemdesign}

\section{Program- og systemtest}
\label{sec:program_og_systemtest}

\section{Brugergrænseflade og interaktionsdesign}

\section{Projektsamarbejdet}
\label{sec:projektsamarbejdet}

\appendix
\section{Versionsstyring}

\section{Changelog}
\begin{tabular}{l l l}
08-04-2015 & CKL & Dokument oprettet \\
14-04-2015 & CKL & Tilføjet use cases og krav \\
\end{tabular}

\section{Timeline}
\begin{tabular}{l l p{8cm}}
11-03-2015 & Første møde med kunde: & Vi har første møde med kunden hvor vi snakker om hvad projektet går ud på og diskuterer krav og løsningsforslag.
\end{tabular}

\bibliographystyle{alpha}
\bibliography{refs}{}

\end{document}
